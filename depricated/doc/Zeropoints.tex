\documentclass{aastex}
\providecommand{\tabularnewline}{\\}
\begin{document}

\section{Photometric Zero Points}

When combining our high-z CSP data with the data from other groups
with the purpose of deriving a color (to constrain reddening, for
example), care must be taken to ensure that the photometric zero-points
are consistent. These zero-points enter our analysis in the cross-filter
k-corrections:

\[
K_{XY}=-2.5\log\left[1+z\right]+2.5\log\left[\frac{\int S_{X}\left(\lambda\right)\Phi\left(\lambda\right)\lambda d\lambda}{\int S_{Y}\left(\lambda/\left(1+z\right)\right)\Phi\left(\lambda\right)\lambda d\lambda}\right]-ZP_{X}+ZP_{Y}\]
where $K_{XY}$ is the k-correction from rest-frame filter X to observed
filter Y, $S_{X}$ and $S_{Y}$ are the transmission functions for
filters X and Y, respectively, $\Phi$ is the SED of the observed
object (type Ia supernova), and $ZP_{X}$ and $ZP_{Y}$ are the photometric
zero-points of filters X and Y, respectively.

If one wishes to construct a color (say $B-V$) from observed filters
X and Y, then due to the cross-filter k-corrections, one must compute
the following term: $\left(ZP_{B}-ZP_{X}\right)-\left(ZP_{V}-ZP_{Y}\right)$.
These zero points can be derived as:\[
ZP_{X}=2.5\log\left[\int S_{X}\left(\lambda\right)F_{o}\left(\lambda\right)\lambda d\lambda\right]+m_{o}\]
where $F_{o}$ is the SED of a fiducial source and $m_{o}$ is the
observed magnitude of this source through filter X. In the case of
the Landolt system, the fiducial source is Vega, whereas in the SDSS
system, there are 4 fiducial WD stars. It is evident from these formulae
that the absolute flux level of the fiducial source is not important,
as we are only dealing with differences in zero-points: only the color
matters. However, this is only true if both photometric systems use
the same fiducial sources. Unfortunately, when combining our NIR data
(for which the ficucial source is Vega) with SDSS data (which is tied
to the AB system), the absolute fluxes of the fiducial sources \emph{do}
matter.

For the purposes of computing photometric zero-points, we have adopted
the SED of Vega as presented in Bohlin \& Gilliland (2004). In the
following subsections, we outline how the zero-points were derived
for each photometric system. Table \ref{tab:zeropoints} shows the
final zero-points adopted.


\subsection{CSP NIR Photometry}

Our YJ photometry uses the standards of Persson et al. (1998), which
are tied to the Elias et al. (1982) standards. In turn, the Elias
standards are ultimately tied to Vega. We are therefore on a Vega
system and use the Bohlin et al. (2004) SED. We assume $J_{vega}=-0.001$
and $Y_{vega}=0.000$ (Stritzinger et al. 2005).


\subsection{Johnson Kron/Cousins BVRI}

Our light-curve templates are based on BVRI photometry as outlined
in Prieto et al. (2006). These templates and calibration of the low-z
sample are based on BVRI photometry from various sources. Still, all
these data are transformed to the standard BVRI system and are therefore
Vega-based. We adopt the following for Vega: $B_{vega}=0.0210$, $V_{vega}=0.0230$,
$R_{vega}=0.0290,$ and $I_{vega}=0.0230$.


\subsection{SNLS griz}

SNLS uses the SDSS filter set, yet they use the Landolt standards,
so are also on a Vega system. Again, we use the Bohlin et al. (2004)
SED and adopt the following magnitudes for Vega: $g_{vega}=0.0220$,
$r_{vega}=0.0281$, $i_{vega}=0.0241$, $z_{vega}=0.0214$ (Alex Conley,
private communication).


\subsection{ESSENCE RI}

Essence uses the Landolt stars as standards and are therefore on a
Vega system. We adopt the same magnitudes for Vega as in the case
of the Johnson filters: $R_{vega}=0.0290$ and $I_{vega}=0.0230$.


\subsection{SDSS II gri}

Unlike all the previous datasets, Sloan uses an AB system that is
tied to the SED of 4 F sub-dwarf stars (Fukugita et al. 1996) whose
absolute fluxes are calibrated by the absolute flux of Vega's continuum.
In order to compute the zero-points, we use the transmission functions
for the APO 2.5m telescope. We then use the transformation equations
from the SDSS website to transform the u'g'r'i'z' magnitudes of the
4 sub-dwarfs to ugriz (2.5 m) values. We also re-calibrate the SEDs
of these 4 sub-dwarfs so that they are consistent with the Bohlin
et al. (2004) SED. We do this by re-binning the Bohlin SED to match
the SED of Vega given by Fukugita et al. (1996), dividing one by the
other, and then multiplying each sub-dwarf SED by the resulting normalization
function. Once this is done, we use the transformed sub-dwarf magnitudes
to compute the zero-points of the filter set.

%
\begin{table}
\begin{tabular}{|c|c|c|}
\hline 
Filter&
Survey&
Zero-point\tabularnewline
\hline
\hline 
Y&
CSP&
12.687\tabularnewline
\hline 
J&
CSP&
12.853\tabularnewline
\hline 
B&
Johnson&
15.289\tabularnewline
\hline 
V&
Johnson&
14.860\tabularnewline
\hline 
R&
Johnson&
15.046\tabularnewline
\hline 
I&
Johnson&
14.546\tabularnewline
\hline 
g&
SNLS&
15.536\tabularnewline
\hline 
r&
SNLS&
14.806\tabularnewline
\hline 
i&
SNLS&
14.554\tabularnewline
\hline 
z&
SNLS&
14.000\tabularnewline
\hline 
g&
SDSS&
14.192\tabularnewline
\hline 
r&
SDSS&
14.220\tabularnewline
\hline 
i&
SDSS&
13.779\tabularnewline
\hline 
R&
Essence&
15.182\tabularnewline
\hline 
I&
Essence&
14.458\tabularnewline
\hline
\end{tabular}


\caption{Photometric zero-points of filter sets used by the high-z CSP. \label{tab:zeropoints}}
\end{table}

\end{document}
